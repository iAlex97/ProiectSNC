\documentclass[12pt,english]{article}
\usepackage[a4paper,bindingoffset=0.2in,%
            left=1in,right=1in,top=1in,bottom=1in,%
            footskip=.25in]{geometry}
\usepackage{amsmath}
\usepackage{graphicx}
\usepackage{listings}
\usepackage{color}

\definecolor{dkgreen}{rgb}{0,0.6,0}
\definecolor{gray}{rgb}{0.5,0.5,0.5}
\definecolor{mauve}{rgb}{0.58,0,0.82}

\lstset{frame=tb,
  language=Matlab,
  aboveskip=3mm,
  belowskip=3mm,
  showstringspaces=false,
  columns=flexible,
  basicstyle={\small\ttfamily},
  numbers=none,
  numberstyle=\tiny\color{gray},
  keywordstyle=\color{blue},
  commentstyle=\color{dkgreen},
  stringstyle=\color{mauve},
  breaklines=true,
  breakatwhitespace=true,
  tabsize=3
}

\title{Tema 2}
\date{2019\\ Octombrie}
\author{Pangratie Andrei - 342 B3}

\begin{document}

\maketitle
\newpage

\tableofcontents
\newpage

\section { Pregatire experiment identificare }
\subsection { Platforma Laborator 3 de citit }
\subsection { Se studiaza fisa de activități }
\subsection { nu exista in cerinta 1.3 ??? }
\subsubsection { Grafice Raspuns indicial (comanda si iesire) }
\subsubsection { [1p] Comentarii referitor grafice obtinute (scris/audio/video - la alegere) }
\subsection { Datele masurate salvate (link ?) }
\subsection { Caracteristici proces }
\subsection { Rezolvare Aplicatii Lab 3 }

\section { REALIZARE SI ANALIZA EXPERIMENT IDENTIFICARE }
\subsection { Expresia Matlab de generare a semnalului SPAB }
\subsection { [2p] Caracteristici semnal SPAB de intrare }
\subsection { [3p] Afisare spectrul semnal SPAB de intrare }
\subsection { [1p] Observatii asupra semnalului SPAB generat }
\subsection { Realizare experiment identificare (conform instructiunilor din laborator) }
\subsection { Fisier rezultate identificare (link). Fiserul este de tip .mat in care este salvat o structura tip iddata care contine intrarea, iesirea si configurarea unor parametri (exp Te). }
\subsection { [3p] Afisare spectrul semnal SPAB de iesire (achizitionat) }
\subsection { [1p] Observatii asupra semnalului achizitonat }

\section { IDENTIFICARE SI VALIDARE MODEL MATLAB }
\subsection { Platforma laborator 4 - citită                                  
\subsection { [2p+1p+2p] Rezolvare Aplicatii Lab 4 
\subsection { Filtrare semnale achiziționate in urma experimentului de identificare. 
\subsubsection { Functii Matlab apelate pentru filtrari: 
\subsubsection { [2p] Spectru semnale filtrate: 
  (comandă și ieșire): 
\subsubsection { [1p] Comentarii asupra spectrului: ............... 
\subsection { Seturile de date de identificare Matlab - iddata pentru identificare si validare. 
\subsubsection { eData  (upload) 
\subsubsection { vData  (upload) 
\subsection { [2p] Estimarea complexității model ARX 
\subsubsection { Utilizare functie advice: ................. 
\subsubsection { Utilizare functie delayest: ................. 
\subsubsection { Estimare complexitate model ARX 
nA
nB
nk
\subsection { Identificare model ARX 
\subsubsection { Descriere model obținut (structură, coeficienți, etc) 
\subsubsection { Valorile funcțiilor criteriu 
\subsubsection { Figurile obținute in urma validării ( resid & compare) (vezi App Laborator 4) 
\subsection { [3p] Identificare model ARMAX 
\subsubsection { Descriere model obținut (structură, coeficienți, etc) 
\subsubsection { Valorile funcțiilor criteriu 
\subsubsection { Figurile obținute in urma validării ( resid & compare) (vezi App Laborator 4) 
\subsection { [3p] Identificare model BJ 
\subsubsection { Descriere model ales (structură, coeficienți, etc) 
\subsubsection { Valorile funcțiilor criteriu 
\subsubsection { Figurile obținute in urma validării ( resid & compare) (vezi App Laborator 4) 
\subsection { [3p] Identificare model OE 
\subsubsection { Descriere model ales (structură, coeficienți, etc) 
\subsubsection { Valorile funcțiilor criteriu 
\subsubsection { Figurile obținute in urma validării ( resid & compare) (vezi App Laborator 4) 
\subsection { Alegere Model Final Matlab 
\subsubsection { Descriere model ales (structură, coeficienți, etc) 
\subsubsection { Valorile funcțiilor criteriu 
\subsubsection { Figurile obținute în urma validării (resid & compare) 
\subsubsection { [1p] Studiul stabilității sistemului: ........... 
\subsubsection { Modelul Matlab ales încărcat este disponibil ?aici?. 
\subsubsection { Comentarii/Observații 

\section { MODELARE SI IDENTIFICARE FOLOSIND WIMPIM }

4.1. [1p] Pregătire date inițiale WINPIM. Fisierul txt obtinut (upload) 
4.2. [1p] Incarcare fisier in WinPIM si specificare perioada de esantionare 
4.3. [1p] Aplicare filtrare set de date (eliminarea componentei continue) 
4.4. [1p] Estimarea comlexitatii
4.5. [5p]  Identificare si validare modele 
4.6. [1p] Modelul ales este anexat aici (BXY.mod) 
4.7. [2p] Simulare model ales WinPIM si simulare model ales Matlab. 
4.8. [2p]  Graficele simularilor sunt disponibile aici .  
4.9. Modelul final ales pentru continuarea proiectului este: 

\section { CALCUL REGULATOR RST-1, SIMULARE SI VALIDARE }

5.1. Platforma laborator 6 - citită 
PROIECTARE REGLARE 
5.2.  Obiective de reglare impuse  : 
5.3. Pulsatia naturala si atenuarea echivalente cu obiectivele de reglare impuse: 
5.4.[2p] Polii dominanti discreti impusi ca urmare a obiectivelor de reglare: 
5.5. [2p] Specificare polinom P: 
5.6. [2p] Grade polinoame ecuatia Sylvester 
5.7. [2p]  Matricea M asociata : 
5.8. [1p] Solutia ecuatiei
5.9. [3p] Pentru ca sistemul sa ofere timp de raspuns minim si suprareglaj < 5% se aleg: 
5.10. [4p]  Simulare sistem in bucla inchisa (comanda, referinta, iesirea), in conditii de perturbatii treapta (25\% amplitudine) aplicate dupa stabilizarea sistemului fata de referinta. Graficele sunt prezentate aici: 
5.11. [1p] Observatii legate de rezultatele obtinute: 
5.12. [1p] Specificare polinom P: 
5.13 [1p] Grade polinoame ecuatia Sylvester 
5.14. [2p]  Matricea M asociata : 
5.15. [1p] Solutia ecuatiei
5.16. [2p]  Simulare sistem in bucla inchisa (comanda, referinta, iesirea), in conditii de perturbatii treapta (25\% amplitudine) aplicate dupa stabilizarea sistemului fata de referinta. Graficele sunt prezentate aici: 
5.17. [1p] Observatii legate de rezultatele obtinute: 

\section { PROIECTARE REGULATOR RST-1 - WINPIM }

6.1. Specificare performante in urmarire respectiv in reglare: 
6.2. Pentru regulatorul calculat folosind metoda Pole Placement, cu integrator, polinoamele R,S,T sunt: 
6.3. Fisierul WinPim cu regulator si model este aici. 
6.4. Simulare sistem in bucla inchisa (comanda, referinta, iesirea), in conditii de perturbatii treapta (25\% amplitudine) aplicate dupa stabilizarea sistemului fata de referinta. Graficele sunt prezentate aici: 
6.5. Observatii legate de rezultatele obtinute: 

\section {}
\subsection {}
\subsubsection {}

\end{document}
